\documentclass{article}
\usepackage{float}
\usepackage{hyperref}
\usepackage[utf8]{inputenc}
\usepackage{graphicx}
\usepackage{caption}
\usepackage{subcaption}
\usepackage{algpseudocode}
\usepackage{algorithm}
\usepackage{mathtools}
\usepackage{amsmath,amssymb}
\usepackage{siunitx}
\usepackage{listings}
\usepackage{bbm}
\usepackage[most]{tcolorbox}
\newcommand{\e}[1]{{\mathbb E}\left[ #1 \right]}
\definecolor{block-gray}{gray}{0.90}
\newtcolorbox{code}{colback=block-gray,grow to right by=-1mm,grow to left by=-1mm,boxrule=0pt,boxsep=0pt,breakable}
\lstset{
  basicstyle=\ttfamily,
  columns=fullflexible,
  frame=single,
  breaklines=true,
  postbreak=\mbox{\textcolor{red}{$\hookrightarrow$}\space},
}

\title{\vspace{-3cm}Charlie meeting 20th June\vspace{-3em}}
\author{}
\date{}

\begin{document}
\maketitle
\section*{Meeting}
In this meeting, Andrew and I discussed the result I produced last week. 
Everything is working, which is good. 
This week I will work on different type of paramtisation to paramatise the energy function, Andrew suggested a few different methods as basis functions, such as polynomial, splines and Fourier representation. 
We also talked about my methods of adding jitter, which might not be entire reasonable so I am fixing that too. 
He also talked about a way that makes this methods scalable, which is that instead of a grid of invariance, we only condition on the local area around the data points as well as the test points we wish to evaluate. 
\end{document}

