\documentclass{article}
\usepackage{float}
\usepackage{hyperref}
\usepackage[utf8]{inputenc}
\usepackage{graphicx}
\usepackage{caption}
\usepackage{subcaption}
\usepackage{algpseudocode}
\usepackage{algorithm}
\usepackage{mathtools}
\usepackage{amsmath,amssymb}
\usepackage{siunitx}
\usepackage{listings}
\usepackage{bbm}
\usepackage[most]{tcolorbox}
\newcommand{\e}[1]{{\mathbb E}\left[ #1 \right]}
\definecolor{block-gray}{gray}{0.90}
\newtcolorbox{code}{colback=block-gray,grow to right by=-1mm,grow to left by=-1mm,boxrule=0pt,boxsep=0pt,breakable}
\lstset{
  basicstyle=\ttfamily,
  columns=fullflexible,
  frame=single,
  breaklines=true,
  postbreak=\mbox{\textcolor{red}{$\hookrightarrow$}\space},
}

\title{\vspace{-3cm}Charlie meeting 20th June\vspace{-3em}}
\author{}
\date{}

\begin{document}
\maketitle
\section*{Meeting}
In this meeting, we talked about how I should change my ODE data generator to a SDE model by adding noise to the dynamics; i.e. by adding noise to every Euler discrete integration steps.
We have also talked about using Laurent Series to represent the inverse term.
We also talked about how to evaluate our metrics, by producing future predictions of the trajectory. 
Or else, I should use a latent Markov GP, using Kalman filter or particle filter model that accounts for the noise instead.
\end{document}

